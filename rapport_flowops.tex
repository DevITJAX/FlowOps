\documentclass[12pt,a4paper]{report}

% ================== PACKAGES ==================
\usepackage[utf8]{inputenc}
\usepackage[T1]{fontenc}
\usepackage[french]{babel}
\usepackage{graphicx}
\usepackage{geometry}
\usepackage{hyperref}
\usepackage{fancyhdr}
\usepackage{titlesec}
\usepackage{tocloft}
\usepackage{array}
\usepackage{booktabs}
\usepackage{caption}
\usepackage{float}
\usepackage{amssymb}
\usepackage{listings}
\usepackage{xcolor}
\usepackage{tabularx}
\usepackage{enumitem}

% ================== CONFIGURATION ==================
\geometry{margin=2.5cm}
\hypersetup{colorlinks=true, linkcolor=blue, urlcolor=blue}
\pagestyle{fancy}
\fancyhf{}
\fancyhead[L]{\leftmark}
\fancyhead[R]{FlowOps}
\fancyfoot[C]{\thepage}

\lstset{
    basicstyle=\ttfamily\small,
    breaklines=true,
    frame=single,
    backgroundcolor=\color{gray!10},
    numbers=left,
    numberstyle=\tiny\color{gray},
    tabsize=2,
    showstringspaces=false
}

\begin{document}

% ======================================================================
%                        PARTIE I : FRONT MATTER
% ======================================================================

% ================== PAGE DE GARDE ==================
\begin{titlepage}
    \centering
    \vspace*{0.5cm}
    
    % Logos
    \begin{minipage}{0.45\textwidth}
        \centering
        \fbox{\parbox{3cm}{\centering\vspace{0.8cm}Logo\\Université\vspace{0.8cm}}}
    \end{minipage}
    \hfill
    \begin{minipage}{0.45\textwidth}
        \centering
        \fbox{\parbox{3cm}{\centering\vspace{0.8cm}Logo\\Entreprise\vspace{0.8cm}}}
    \end{minipage}
    
    \vspace{1.5cm}
    
    {\scshape\LARGE [Nom de l'Université / École] \par}
    \vspace{0.3cm}
    {\scshape\large Département Informatique \par}
    {\scshape\large Filière : Génie Logiciel \par}
    
    \vspace{1.5cm}
    
    \rule{\textwidth}{1.5pt}
    \vspace{0.4cm}
    {\Huge\bfseries Rapport de Projet de Fin d'Études \par}
    \vspace{0.3cm}
    {\LARGE\bfseries FlowOps \par}
    \vspace{0.2cm}
    {\Large Application de Gestion de Projets avec Pipeline CI/CD \par}
    \vspace{0.4cm}
    \rule{\textwidth}{1.5pt}
    
    \vspace{1.5cm}
    
    \begin{tabular}{rl}
        \textbf{Réalisé par :} & [Prénom NOM] \\[0.2cm]
        \textbf{Encadrant académique :} & [Prénom NOM] \\[0.2cm]
        \textbf{Encadrant professionnel :} & [Prénom NOM] \\[0.2cm]
        \textbf{Période de stage :} & [Date début] -- [Date fin] \\
    \end{tabular}
    
    \vfill
    {\large Année Universitaire 2024 -- 2025 \par}
\end{titlepage}

% ================== DÉDICACES ==================
\chapter*{Dédicaces}
\addcontentsline{toc}{chapter}{Dédicaces}
\vspace{2cm}
\begin{flushright}
\textit{À mes parents,\\
pour leur soutien inconditionnel...\\[0.5cm]
À mes enseignants,\\
pour leur accompagnement...\\[0.5cm]
À tous ceux qui m'ont soutenu...}
\end{flushright}

% ================== REMERCIEMENTS ==================
\chapter*{Remerciements}
\addcontentsline{toc}{chapter}{Remerciements}

Je tiens à exprimer ma profonde gratitude à toutes les personnes qui ont contribué à la réussite de ce projet.

Mes remerciements s'adressent tout d'abord à \textbf{[Nom de l'encadrant académique]}, mon encadrant pédagogique, pour ses conseils avisés et son suivi rigoureux.

Je remercie également \textbf{[Nom de l'encadrant professionnel]} pour son accompagnement technique et sa disponibilité tout au long de ce stage.

J'exprime ma reconnaissance à l'ensemble du corps enseignant de [Nom de l'établissement] pour la qualité de la formation dispensée.

Enfin, je remercie ma famille pour son soutien moral constant.

% ================== RÉSUMÉ ==================
\chapter*{Résumé}
\addcontentsline{toc}{chapter}{Résumé}

Dans le cadre de mon Projet de Fin d'Études, j'ai conçu et développé \textbf{FlowOps}, une application web de gestion de projets intégrant les pratiques DevOps modernes.

\textbf{Contexte :} Les équipes de développement font face à des défis croissants en termes de coordination, suivi des tâches et automatisation du déploiement.

\textbf{Objectifs :} Concevoir une solution complète permettant la gestion des projets, tâches et sprints, avec un pipeline CI/CD automatisé.

\textbf{Réalisation :} L'application a été développée avec la stack MERN (MongoDB, Express.js, React, Node.js), conteneurisée avec Docker, et déployée sur Microsoft Azure via GitHub Actions.

\textbf{Résultats :} Un système fonctionnel avec authentification JWT, tableau Kanban, gestion des sprints, et déploiement automatisé à chaque commit.

\vspace{0.5cm}
\textbf{Mots-clés :} Gestion de projets, DevOps, CI/CD, MERN Stack, Docker, Azure, GitHub Actions.

% ================== ABSTRACT ==================
\chapter*{Abstract}
\addcontentsline{toc}{chapter}{Abstract}

As part of my Final Year Project, I designed and developed \textbf{FlowOps}, a project management web application integrating modern DevOps practices.

\textbf{Context:} Development teams face growing challenges in coordination, task tracking, and deployment automation.

\textbf{Objectives:} Design a comprehensive solution for managing projects, tasks, and sprints, with an automated CI/CD pipeline.

\textbf{Implementation:} The application was built using the MERN stack, containerized with Docker, and deployed to Microsoft Azure via GitHub Actions.

\textbf{Results:} A functional system with JWT authentication, Kanban board, sprint management, and automated deployment on each commit.

\vspace{0.5cm}
\textbf{Keywords:} Project Management, DevOps, CI/CD, MERN Stack, Docker, Azure, GitHub Actions.

% ================== TABLES ==================
\tableofcontents
\listoffigures
\listoftables

% ================== LISTE DES ABRÉVIATIONS ==================
\chapter*{Liste des Abréviations}
\addcontentsline{toc}{chapter}{Liste des Abréviations}

\begin{tabular}{ll}
\textbf{API} & Application Programming Interface \\
\textbf{ACR} & Azure Container Registry \\
\textbf{CD} & Continuous Deployment \\
\textbf{CI} & Continuous Integration \\
\textbf{CRUD} & Create, Read, Update, Delete \\
\textbf{JWT} & JSON Web Token \\
\textbf{MERN} & MongoDB, Express, React, Node.js \\
\textbf{PFE} & Projet de Fin d'Études \\
\textbf{REST} & Representational State Transfer \\
\textbf{SPA} & Single Page Application \\
\textbf{UML} & Unified Modeling Language \\
\end{tabular}

% ======================================================================
%                    PARTIE II : CONTEXTE ORGANISATIONNEL
% ======================================================================

\chapter{Contexte Organisationnel}

\section{Introduction}

Ce chapitre présente le cadre organisationnel dans lequel s'inscrit ce projet de fin d'études. Nous présenterons l'organisme d'accueil (si applicable) ainsi que le contexte général du projet.

\section{Présentation de l'Organisme d'Accueil}

\subsection{Présentation Générale}

[Décrire l'entreprise/organisation d'accueil : historique, secteur d'activité, taille, implantations géographiques]

\begin{figure}[H]
\centering
\fbox{\parbox{0.6\textwidth}{\centering\vspace{1.5cm}Logo de l'Entreprise\vspace{1.5cm}}}
\caption{Logo de [Nom de l'entreprise]}
\end{figure}

\subsection{Organigramme}

[Présenter la structure organisationnelle et positionner votre équipe/département]

\begin{figure}[H]
\centering
\fbox{\parbox{0.85\textwidth}{\centering\vspace{2cm}Organigramme de l'Entreprise\vspace{2cm}}}
\caption{Organigramme de [Nom de l'entreprise]}
\end{figure}

\subsection{Équipe de Projet}

L'équipe au sein de laquelle j'ai effectué mon stage était composée de :

\begin{itemize}
    \item [Rôle 1] : responsable de...
    \item [Rôle 2] : en charge de...
    \item Moi-même : développeur full-stack / DevOps
\end{itemize}

\section{Conclusion}

Ce chapitre a permis de situer le contexte organisationnel du projet. Le chapitre suivant présentera le cadre général et la méthodologie adoptée.

% ======================================================================
%                    PARTIE III : CADRE GÉNÉRAL DU PROJET
% ======================================================================

\chapter{Cadre Général du Projet}

\section{Introduction}

Ce chapitre présente le cadre général du projet : la problématique, les objectifs, les livrables attendus ainsi que la méthodologie de travail adoptée.

\section{Problématique}

Les équipes de développement modernes font face à plusieurs défis :

\begin{itemize}
    \item \textbf{Coordination} : Difficulté à suivre l'avancement des tâches
    \item \textbf{Visibilité} : Manque de clarté sur les priorités et les deadlines
    \item \textbf{Déploiement} : Processus manuels, lents et sources d'erreurs
    \item \textbf{Collaboration} : Communication fragmentée entre les membres
\end{itemize}

\textbf{Problématique centrale :} Comment concevoir une application de gestion de projets intégrant les meilleures pratiques DevOps pour automatiser le cycle de vie applicatif ?

\section{Objectifs du Projet}

\subsection{Objectifs Fonctionnels}

\begin{itemize}
    \item Permettre la gestion des utilisateurs avec différents rôles
    \item Offrir des fonctionnalités CRUD pour les projets et tâches
    \item Implémenter un tableau Kanban interactif
    \item Gérer les sprints selon la méthodologie Scrum
    \item Fournir un journal d'activités et des notifications
\end{itemize}

\subsection{Objectifs Techniques}

\begin{itemize}
    \item Concevoir une architecture full-stack moderne
    \item Sécuriser l'application avec JWT
    \item Conteneuriser avec Docker
    \item Automatiser le déploiement avec GitHub Actions
    \item Déployer sur Microsoft Azure
\end{itemize}

\section{Livrables}

\begin{table}[H]
\centering
\caption{Liste des livrables}
\begin{tabular}{|l|l|}
\hline
\textbf{Livrable} & \textbf{Description} \\
\hline
Application Web & Frontend React + Backend Express.js \\
Base de données & Schéma MongoDB \\
Pipeline CI/CD & Workflows GitHub Actions \\
Images Docker & Backend et Frontend containerisés \\
Documentation & Rapport technique et guide utilisateur \\
\hline
\end{tabular}
\end{table}

\section{Méthodologie de Travail}

\subsection{Choix de la Méthodologie Agile}

Le projet a été réalisé selon la méthodologie \textbf{Agile Scrum}, caractérisée par :

\begin{itemize}
    \item Des itérations courtes (sprints de 2 semaines)
    \item Des réunions régulières (daily standups)
    \item Une adaptation continue aux changements
    \item Une livraison incrémentale de valeur
\end{itemize}

\subsection{Cérémonies Scrum}

\begin{table}[H]
\centering
\caption{Cérémonies Scrum appliquées}
\begin{tabular}{|l|l|l|}
\hline
\textbf{Cérémonie} & \textbf{Fréquence} & \textbf{Objectif} \\
\hline
Sprint Planning & Début de sprint & Définir le backlog du sprint \\
Daily Standup & Quotidien & Synchronisation de l'équipe \\
Sprint Review & Fin de sprint & Démonstration du travail \\
Rétrospective & Fin de sprint & Amélioration continue \\
\hline
\end{tabular}
\end{table}

\subsection{Outils Utilisés}

\begin{table}[H]
\centering
\caption{Outils de gestion de projet}
\begin{tabular}{|l|l|}
\hline
\textbf{Outil} & \textbf{Usage} \\
\hline
GitHub Projects & Suivi des tâches et Kanban \\
Git & Gestion de version \\
Visual Studio Code & IDE de développement \\
Postman & Test des APIs \\
Discord/Teams & Communication \\
\hline
\end{tabular}
\end{table}

\section{Planning du Projet}

\begin{figure}[H]
\centering
\fbox{\parbox{0.95\textwidth}{\centering\vspace{2cm}Diagramme de Gantt du Projet\vspace{2cm}}}
\caption{Planning prévisionnel du projet}
\end{figure}

\section{Conclusion}

Ce chapitre a présenté le cadre méthodologique du projet. Le chapitre suivant détaillera l'analyse des besoins et les spécifications.

% ======================================================================
%                    PARTIE IV : ANALYSE ET SPÉCIFICATIONS
% ======================================================================

\chapter{Analyse et Spécifications}

\section{Introduction}

Ce chapitre présente l'analyse des besoins fonctionnels et non-fonctionnels, ainsi que l'identification des acteurs du système.

\section{Étude de l'Existant}

\subsection{Solutions Existantes}

Plusieurs solutions de gestion de projets existent sur le marché :

\begin{table}[H]
\centering
\caption{Analyse des solutions existantes}
\begin{tabular}{|l|p{4cm}|p{4cm}|}
\hline
\textbf{Solution} & \textbf{Points Forts} & \textbf{Points Faibles} \\
\hline
Jira & Complet, Agile-ready & Complexe, coûteux \\
Trello & Simple, intuitif & Limité pour Scrum \\
Asana & Polyvalent & Pas DevOps-oriented \\
\hline
\end{tabular}
\end{table}

\subsection{Critique de l'Existant}

Les solutions existantes présentent des limitations :
\begin{itemize}
    \item Dépendance à des solutions propriétaires
    \item Coût élevé pour les fonctionnalités avancées
    \item Intégration DevOps limitée ou complexe
    \item Impossibilité d'auto-hébergement
\end{itemize}

\section{Identification des Acteurs}

\begin{table}[H]
\centering
\caption{Acteurs du système}
\begin{tabularx}{\textwidth}{|l|X|}
\hline
\textbf{Acteur} & \textbf{Description} \\
\hline
Administrateur & Gère les utilisateurs, paramètres globaux, accès complet \\
Chef de Projet & Crée/gère projets et sprints, assigne les tâches \\
Membre & Consulte les projets assignés, met à jour ses tâches \\
Système CI/CD & Déclenche automatiquement build et déploiement \\
\hline
\end{tabularx}
\end{table}

\section{Exigences Fonctionnelles}

\begin{table}[H]
\centering
\caption{Exigences fonctionnelles}
\begin{tabular}{|l|p{8cm}|}
\hline
\textbf{ID} & \textbf{Description} \\
\hline
EF01 & Le système doit permettre l'inscription et la connexion \\
EF02 & Le système doit gérer les rôles (Admin, PM, Member) \\
EF03 & Le système doit permettre le CRUD des projets \\
EF04 & Le système doit permettre le CRUD des tâches \\
EF05 & Le système doit afficher un tableau Kanban \\
EF06 & Le système doit gérer les sprints \\
EF07 & Le système doit enregistrer les activités \\
EF08 & Le système doit envoyer des notifications \\
\hline
\end{tabular}
\end{table}

\section{Exigences Non-Fonctionnelles}

\begin{table}[H]
\centering
\caption{Exigences non-fonctionnelles}
\begin{tabular}{|l|p{8cm}|}
\hline
\textbf{Catégorie} & \textbf{Exigence} \\
\hline
Sécurité & Authentification JWT, mots de passe hashés \\
Performance & Temps de réponse < 2 secondes \\
Disponibilité & Uptime > 99\% \\
Maintenabilité & Code modulaire, documenté \\
Portabilité & Conteneurs Docker multi-plateforme \\
Scalabilité & Architecture cloud-ready \\
\hline
\end{tabular}
\end{table}

\section{Règles de Gestion}

\begin{itemize}
    \item RG01 : Un utilisateur doit avoir un email unique
    \item RG02 : Un projet doit avoir au moins un propriétaire
    \item RG03 : Une tâche appartient à un seul projet
    \item RG04 : Un sprint appartient à un seul projet
    \item RG05 : Seul le propriétaire peut supprimer un projet
\end{itemize}

\section{Conclusion}

Ce chapitre a permis de formaliser les besoins du système. Le chapitre suivant présentera la conception détaillée.

% ======================================================================
%                    PARTIE V : CONCEPTION
% ======================================================================

\chapter{Conception}

\section{Introduction}

Ce chapitre présente la conception du système FlowOps à travers les différents diagrammes UML : cas d'utilisation, classes, séquences et activités.

\section{Diagramme de Cas d'Utilisation}

\begin{figure}[H]
\centering
\fbox{\parbox{0.95\textwidth}{\centering\vspace{4cm}
\textbf{Diagramme de Cas d'Utilisation Global}\\[0.5cm]
Acteurs : Administrateur, Chef de Projet, Membre\\[0.3cm]
Cas principaux :\\
- S'authentifier / S'inscrire\\
- Gérer les projets\\
- Gérer les tâches\\
- Gérer les sprints\\
- Visualiser le Kanban\\
- Gérer les utilisateurs (Admin)
\vspace{2cm}}}
\caption{Diagramme de cas d'utilisation global}
\end{figure}

\subsection{Description des Cas d'Utilisation}

\textbf{CU01 - S'authentifier}
\begin{itemize}
    \item \textbf{Acteur} : Tous les utilisateurs
    \item \textbf{Précondition} : Posséder un compte
    \item \textbf{Scénario} : Saisir credentials → Validation → Token JWT → Accès
    \item \textbf{Postcondition} : Utilisateur authentifié
\end{itemize}

\textbf{CU02 - Gérer les tâches}
\begin{itemize}
    \item \textbf{Acteur} : Membre, Chef de Projet
    \item \textbf{Précondition} : Être membre du projet
    \item \textbf{Scénario} : Créer/Modifier/Supprimer/Déplacer tâche
    \item \textbf{Postcondition} : Tâche mise à jour
\end{itemize}

\section{Diagramme de Classes}

\begin{figure}[H]
\centering
\fbox{\parbox{0.95\textwidth}{\centering\vspace{5cm}
\textbf{Diagramme de Classes}\\[0.5cm]
\textbf{User} : id, name, email, password, role\\
\textbf{Project} : id, name, description, status, owner, members[]\\
\textbf{Task} : id, title, status, priority, assignee, project, sprint\\
\textbf{Sprint} : id, name, startDate, endDate, status, project\\
\textbf{Comment} : id, content, author, task\\
\textbf{Activity} : id, action, user, target, createdAt\\[0.3cm]
Relations : Project 1--* Task, Project 1--* Sprint, Task *--1 User
\vspace{2cm}}}
\caption{Diagramme de classes}
\end{figure}

\section{Diagrammes de Séquence}

\subsection{Séquence : Authentification}

\begin{figure}[H]
\centering
\fbox{\parbox{0.95\textwidth}{\centering\vspace{3cm}
\textbf{Diagramme de Séquence - Authentification}\\[0.3cm]
User → Frontend : Saisie credentials\\
Frontend → Backend : POST /api/auth/login\\
Backend → MongoDB : findOne(email)\\
Backend → Backend : bcrypt.compare()\\
Backend → Backend : jwt.sign()\\
Backend → Frontend : {token, user}\\
Frontend → LocalStorage : store(token)
\vspace{2cm}}}
\caption{Diagramme de séquence : authentification}
\end{figure}

\subsection{Séquence : Création de Tâche}

\begin{figure}[H]
\centering
\fbox{\parbox{0.95\textwidth}{\centering\vspace{3cm}
\textbf{Diagramme de Séquence - Création Tâche}\\[0.3cm]
User → Frontend : Formulaire tâche\\
Frontend → Backend : POST /api/projects/:id/tasks + JWT\\
Backend → Middleware : verifyToken()\\
Backend → MongoDB : Task.create()\\
Backend → MongoDB : Activity.create()\\
Backend → Frontend : {task} 201
\vspace{2cm}}}
\caption{Diagramme de séquence : création d'une tâche}
\end{figure}

\subsection{Séquence : Pipeline CI/CD}

\begin{figure}[H]
\centering
\fbox{\parbox{0.95\textwidth}{\centering\vspace{3cm}
\textbf{Diagramme de Séquence - Pipeline CI/CD}\\[0.3cm]
Dev → GitHub : git push main\\
GitHub → Actions : trigger workflow\\
Actions → Docker : build images\\
Actions → ACR : push images\\
Actions → Azure : deploy containers\\
Azure → ACR : pull images\\
Azure → Azure : start containers
\vspace{2cm}}}
\caption{Diagramme de séquence : pipeline CI/CD}
\end{figure}

\section{Diagramme d'Activité}

\begin{figure}[H]
\centering
\fbox{\parbox{0.95\textwidth}{\centering\vspace{3cm}
\textbf{Diagramme d'Activité - Gestion d'une Tâche}\\[0.3cm]
[Début] → Créer tâche → Assigner → [ToDo]\\
→ Commencer travail → [In Progress]\\
→ Terminer → [Review] → Valider → [Done]\\
→ [Fin]
\vspace{2cm}}}
\caption{Diagramme d'activité : cycle de vie d'une tâche}
\end{figure}

\section{Modèle de Données}

\begin{table}[H]
\centering
\caption{Structure des collections MongoDB}
\begin{tabular}{|l|l|p{6cm}|}
\hline
\textbf{Collection} & \textbf{Champs Clés} & \textbf{Relations} \\
\hline
users & id, name, email, role & - \\
projects & id, name, owner, members & owner → users, members → users[] \\
tasks & id, title, status, priority & project → projects, assignee → users \\
sprints & id, name, dates, status & project → projects \\
comments & id, content, author & task → tasks, author → users \\
activities & id, action, details & user → users \\
\hline
\end{tabular}
\end{table}

\section{Conclusion}

Ce chapitre a présenté la conception UML du système. Le chapitre suivant détaillera l'architecture technique.

% ======================================================================
%                    PARTIE VI : ARCHITECTURE TECHNIQUE
% ======================================================================

\chapter{Architecture Technique}

\section{Introduction}

Ce chapitre présente l'architecture technique du système, les choix technologiques et leur justification.

\section{Architecture Globale}

\begin{figure}[H]
\centering
\fbox{\parbox{0.95\textwidth}{\centering\vspace{3cm}
\textbf{Architecture Trois Tiers}\\[0.5cm]
\textbf{Présentation} : React + Vite + Nginx\\
$\downarrow$ HTTPS / REST API\\
\textbf{Métier} : Node.js + Express.js\\
$\downarrow$ MongoDB Protocol\\
\textbf{Données} : Azure Cosmos DB (MongoDB)
\vspace{2cm}}}
\caption{Architecture trois tiers}
\end{figure}

\section{Architecture de Déploiement}

\begin{figure}[H]
\centering
\fbox{\parbox{0.95\textwidth}{\centering\vspace{3cm}
\textbf{Architecture Cloud Azure}\\[0.5cm]
GitHub Repository\\
$\downarrow$ GitHub Actions\\
Azure Container Registry\\
$\downarrow$\\
App Service (Backend) + App Service (Frontend)\\
$\leftrightarrow$ Azure Cosmos DB
\vspace{2cm}}}
\caption{Architecture de déploiement Azure}
\end{figure}

\section{Choix Technologiques}

\subsection{Stack MERN}

\begin{table}[H]
\centering
\caption{Technologies de la stack MERN}
\begin{tabular}{|l|l|p{6cm}|}
\hline
\textbf{Technologie} & \textbf{Version} & \textbf{Justification} \\
\hline
MongoDB & 7.x & Flexibilité NoSQL, scalabilité \\
Express.js & 4.x & Framework léger, middleware \\
React & 19.x & UI réactives, composants \\
Node.js & 20 LTS & JavaScript côté serveur, npm \\
\hline
\end{tabular}
\end{table}

\subsection{Technologies Complémentaires}

\begin{table}[H]
\centering
\caption{Technologies complémentaires}
\begin{tabular}{|l|l|p{5cm}|}
\hline
\textbf{Technologie} & \textbf{Rôle} & \textbf{Justification} \\
\hline
Docker & Conteneurisation & Portabilité, isolation \\
Nginx & Serveur web & Performance, reverse proxy \\
JWT & Authentification & Stateless, sécurisé \\
Vite & Build frontend & Rapidité, hot reload \\
Bootstrap & UI Framework & Responsive, composants \\
\hline
\end{tabular}
\end{table}

\subsection{Services Azure}

\begin{table}[H]
\centering
\caption{Services Azure utilisés}
\begin{tabular}{|l|p{7cm}|}
\hline
\textbf{Service} & \textbf{Usage} \\
\hline
Azure App Service & Hébergement des conteneurs (Backend + Frontend) \\
Azure Container Registry & Stockage des images Docker \\
Azure Cosmos DB & Base de données MongoDB managée \\
\hline
\end{tabular}
\end{table}

\section{Outils de Développement}

\begin{table}[H]
\centering
\caption{Outils de développement}
\begin{tabular}{|l|l|}
\hline
\textbf{Outil} & \textbf{Usage} \\
\hline
Visual Studio Code & IDE principal \\
Git & Gestion de version \\
GitHub & Hébergement du code + CI/CD \\
Postman & Test des APIs \\
Docker Desktop & Développement local \\
Azure CLI & Gestion des ressources cloud \\
\hline
\end{tabular}
\end{table}

\section{Conclusion}

Ce chapitre a présenté l'architecture technique et justifié les choix. Le chapitre suivant détaillera la réalisation.

% ======================================================================
%                    PARTIE VII : RÉALISATION
% ======================================================================

\chapter{Réalisation}

\section{Introduction}

Ce chapitre présente la réalisation technique du projet, organisée par modules fonctionnels, avec les captures d'écran des interfaces principales.

\section{Module Authentification}

\subsection{Fonctionnalités}
\begin{itemize}
    \item Inscription avec validation des données
    \item Connexion avec génération de token JWT
    \item Protection des routes avec middleware
\end{itemize}

\subsection{Interface}

\begin{figure}[H]
\centering
\fbox{\parbox{0.90\textwidth}{\centering\vspace{3cm}Capture - Page de Connexion\vspace{3cm}}}
\caption{Page de connexion}
\end{figure}

\begin{figure}[H]
\centering
\fbox{\parbox{0.90\textwidth}{\centering\vspace{3cm}Capture - Page d'Inscription\vspace{3cm}}}
\caption{Page d'inscription}
\end{figure}

\section{Module Tableau de Bord}

\subsection{Fonctionnalités}
\begin{itemize}
    \item Vue d'ensemble des projets
    \item Statistiques et indicateurs
    \item Accès rapide aux tâches récentes
\end{itemize}

\begin{figure}[H]
\centering
\fbox{\parbox{0.90\textwidth}{\centering\vspace{3cm}Capture - Tableau de Bord\vspace{3cm}}}
\caption{Tableau de bord principal}
\end{figure}

\section{Module Gestion des Projets}

\subsection{Fonctionnalités}
\begin{itemize}
    \item Liste des projets avec filtres
    \item Création/modification de projets
    \item Gestion des membres d'équipe
\end{itemize}

\begin{figure}[H]
\centering
\fbox{\parbox{0.90\textwidth}{\centering\vspace{3cm}Capture - Liste des Projets\vspace{3cm}}}
\caption{Liste des projets}
\end{figure}

\begin{figure}[H]
\centering
\fbox{\parbox{0.90\textwidth}{\centering\vspace{3cm}Capture - Détail d'un Projet\vspace{3cm}}}
\caption{Page de détail d'un projet}
\end{figure}

\section{Module Gestion des Tâches (Kanban)}

\subsection{Fonctionnalités}
\begin{itemize}
    \item Tableau Kanban avec colonnes (To Do, In Progress, Done)
    \item Drag and drop pour changer le statut
    \item Filtrage par assigné, priorité, sprint
    \item Modal de détail avec commentaires
\end{itemize}

\begin{figure}[H]
\centering
\fbox{\parbox{0.90\textwidth}{\centering\vspace{3cm}Capture - Tableau Kanban\vspace{3cm}}}
\caption{Tableau Kanban des tâches}
\end{figure}

\begin{figure}[H]
\centering
\fbox{\parbox{0.90\textwidth}{\centering\vspace{3cm}Capture - Détail d'une Tâche\vspace{3cm}}}
\caption{Modal de détail d'une tâche}
\end{figure}

\section{Module Gestion des Sprints}

\subsection{Fonctionnalités}
\begin{itemize}
    \item Création de sprints avec dates
    \item Association des tâches aux sprints
    \item Démarrage et clôture des sprints
\end{itemize}

\begin{figure}[H]
\centering
\fbox{\parbox{0.90\textwidth}{\centering\vspace{3cm}Capture - Gestion des Sprints\vspace{3cm}}}
\caption{Liste des sprints}
\end{figure}

\section{Module Administration}

\subsection{Fonctionnalités}
\begin{itemize}
    \item Gestion des utilisateurs
    \item Attribution des rôles
    \item Paramètres système
\end{itemize}

\begin{figure}[H]
\centering
\fbox{\parbox{0.90\textwidth}{\centering\vspace{3cm}Capture - Administration\vspace{3cm}}}
\caption{Page d'administration des utilisateurs}
\end{figure}

\section{Conclusion}

Ce chapitre a présenté les modules fonctionnels réalisés. Le chapitre suivant détaillera la partie DevOps et CI/CD.

% ======================================================================
%                    PARTIE VIII : DEVOPS ET CI/CD
% ======================================================================

\chapter{DevOps et CI/CD}

\section{Introduction}

Ce chapitre présente la mise en œuvre des pratiques DevOps : conteneurisation, automatisation et déploiement cloud.

\section{Conteneurisation Docker}

\subsection{Dockerfile Backend}

\begin{lstlisting}[language=Dockerfile, caption=Dockerfile Backend]
FROM node:20-alpine
WORKDIR /app
COPY package*.json ./
RUN npm ci --only=production
COPY . .
EXPOSE 3001
CMD ["node", "server.js"]
\end{lstlisting}

\subsection{Dockerfile Frontend (Multi-stage)}

\begin{lstlisting}[language=Dockerfile, caption=Dockerfile Frontend]
# Build
FROM node:20-alpine AS build
WORKDIR /app
ARG VITE_API_URL
ENV VITE_API_URL=$VITE_API_URL
COPY package*.json ./
RUN npm ci
COPY . .
RUN npm run build

# Production
FROM nginx:alpine
COPY --from=build /app/dist /usr/share/nginx/html
COPY nginx.conf /etc/nginx/conf.d/default.conf
EXPOSE 80
CMD ["nginx", "-g", "daemon off;"]
\end{lstlisting}

\section{Pipeline CI/CD}

\subsection{Architecture du Pipeline}

\begin{figure}[H]
\centering
\fbox{\parbox{0.95\textwidth}{\centering\vspace{2cm}
\textbf{Pipeline CI/CD avec GitHub Actions}\\[0.5cm]
git push → GitHub Actions → Docker Build → Push ACR → Deploy Azure
\vspace{1.5cm}}}
\caption{Architecture du pipeline CI/CD}
\end{figure}

\subsection{Workflow CD}

\begin{lstlisting}[language=yaml, caption=Workflow CD (cd.yml)]
name: CD Pipeline
on:
  push:
    branches: [main]

jobs:
  build-and-deploy:
    runs-on: ubuntu-latest
    steps:
      - uses: actions/checkout@v4
      - uses: docker/setup-buildx-action@v3
      
      - name: Login ACR
        uses: docker/login-action@v3
        with:
          registry: flowopsacr.azurecr.io
          username: ${{ secrets.ACR_USERNAME }}
          password: ${{ secrets.ACR_PASSWORD }}

      - name: Build/Push Backend
        uses: docker/build-push-action@v5
        with:
          context: ./backend
          push: true
          tags: flowopsacr.azurecr.io/flowops-backend:${{ github.sha }}

      - name: Build/Push Frontend
        uses: docker/build-push-action@v5
        with:
          context: ./frontend
          push: true
          build-args: VITE_API_URL=https://flowops-backend.azurewebsites.net/api
          tags: flowopsacr.azurecr.io/flowops-frontend:${{ github.sha }}

      - name: Deploy Backend
        uses: azure/webapps-deploy@v3
        with:
          app-name: flowops-backend
          images: flowopsacr.azurecr.io/flowops-backend:${{ github.sha }}
\end{lstlisting}

\subsection{Exécution du Pipeline}

\begin{figure}[H]
\centering
\fbox{\parbox{0.90\textwidth}{\centering\vspace{2.5cm}Capture - GitHub Actions en cours d'exécution\vspace{2.5cm}}}
\caption{Exécution réussie du pipeline CD}
\end{figure}

\section{Déploiement Azure}

\subsection{Ressources Créées}

\begin{table}[H]
\centering
\caption{Ressources Azure déployées}
\begin{tabular}{|l|l|l|}
\hline
\textbf{Ressource} & \textbf{Nom} & \textbf{Rôle} \\
\hline
Resource Group & flowops-rg & Conteneur logique \\
Container Registry & flowopsacr & Images Docker \\
App Service & flowops-backend & API REST \\
App Service & flowops-frontend & Interface web \\
Cosmos DB & flowops-mongodb & Base de données \\
\hline
\end{tabular}
\end{table}

\subsection{Configuration}

\begin{figure}[H]
\centering
\fbox{\parbox{0.90\textwidth}{\centering\vspace{2.5cm}Capture - Portail Azure\vspace{2.5cm}}}
\caption{Portail Azure - Groupe de ressources FlowOps}
\end{figure}

\begin{figure}[H]
\centering
\fbox{\parbox{0.90\textwidth}{\centering\vspace{2.5cm}Capture - Azure Container Registry\vspace{2.5cm}}}
\caption{Azure Container Registry avec les images}
\end{figure}

\section{Secrets et Sécurité}

\begin{table}[H]
\centering
\caption{Secrets GitHub configurés}
\begin{tabular}{|l|l|}
\hline
\textbf{Secret} & \textbf{Description} \\
\hline
ACR\_USERNAME & Identifiant Azure Container Registry \\
ACR\_PASSWORD & Mot de passe ACR \\
AZURE\_BACKEND\_PUBLISH\_PROFILE & Profil de déploiement backend \\
AZURE\_FRONTEND\_PUBLISH\_PROFILE & Profil de déploiement frontend \\
\hline
\end{tabular}
\end{table}

\section{Conclusion}

Ce chapitre a présenté l'implémentation DevOps. Le chapitre suivant présentera les tests et la validation.

% ======================================================================
%                    PARTIE IX : TESTS ET VALIDATION
% ======================================================================

\chapter{Tests et Validation}

\section{Introduction}

Ce chapitre présente la stratégie de test et les résultats de validation du système.

\section{Stratégie de Test}

\subsection{Types de Tests}

\begin{table}[H]
\centering
\caption{Stratégie de test}
\begin{tabular}{|l|p{7cm}|}
\hline
\textbf{Type} & \textbf{Description} \\
\hline
Tests unitaires & Validation des fonctions individuelles \\
Tests d'intégration & Vérification des interactions API \\
Tests fonctionnels & Validation des cas d'utilisation \\
Tests de santé & Monitoring des endpoints \\
\hline
\end{tabular}
\end{table}

\section{Tests de l'API}

\subsection{Health Check}

\begin{lstlisting}[language=bash, caption=Test du Health Check]
$ curl https://flowops-backend.azurewebsites.net/health

{"status":"ok","timestamp":"2025-12-24T19:18:39.591Z"}
\end{lstlisting}

\subsection{Tests des Endpoints}

\begin{figure}[H]
\centering
\fbox{\parbox{0.90\textwidth}{\centering\vspace{2cm}Capture - Test API avec Postman\vspace{2cm}}}
\caption{Tests des endpoints avec Postman}
\end{figure}

\section{Validation Fonctionnelle}

\begin{table}[H]
\centering
\caption{Checklist de validation}
\begin{tabular}{|l|c|c|}
\hline
\textbf{Fonctionnalité} & \textbf{Résultat} & \textbf{Statut} \\
\hline
Inscription utilisateur & Succès & \checkmark \\
Connexion / Déconnexion & Succès & \checkmark \\
CRUD Projets & Succès & \checkmark \\
CRUD Tâches & Succès & \checkmark \\
Tableau Kanban & Succès & \checkmark \\
Gestion des Sprints & Succès & \checkmark \\
Pipeline CI/CD & Succès & \checkmark \\
Déploiement Azure & Succès & \checkmark \\
\hline
\end{tabular}
\end{table}

\section{URLs de Production}

\begin{table}[H]
\centering
\caption{URLs de l'application déployée}
\begin{tabular}{|l|l|}
\hline
\textbf{Service} & \textbf{URL} \\
\hline
Frontend & https://flowops-frontend.azurewebsites.net \\
Backend API & https://flowops-backend.azurewebsites.net \\
Health Check & https://flowops-backend.azurewebsites.net/health \\
\hline
\end{tabular}
\end{table}

\section{Conclusion}

Les tests ont validé le bon fonctionnement du système. Le chapitre suivant présente la conclusion générale.

% ======================================================================
%                    PARTIE X : CONCLUSION
% ======================================================================

\chapter{Conclusion Générale}

\section{Synthèse du Projet}

Ce projet de fin d'études a permis de concevoir et développer \textbf{FlowOps}, une application complète de gestion de projets intégrant les pratiques DevOps modernes.

\textbf{Réalisations principales :}
\begin{itemize}
    \item Application full-stack avec la stack MERN
    \item Interface utilisateur React avec tableau Kanban
    \item API REST sécurisée avec authentification JWT
    \item Conteneurisation Docker multi-stage
    \item Pipeline CI/CD automatisé avec GitHub Actions
    \item Déploiement cloud sur Microsoft Azure
\end{itemize}

\section{Difficultés Rencontrées}

\begin{itemize}
    \item Configuration du pipeline CI/CD pour les conteneurs
    \item Gestion des variables d'environnement entre environnements
    \item Configuration de Azure Cosmos DB avec MongoDB API
    \item Résolution des problèmes de démarrage des conteneurs
    \item Intégration du frontend avec le backend en production
\end{itemize}

\section{Solutions Apportées}

\begin{itemize}
    \item Utilisation de build args Docker pour les URLs d'API
    \item Configuration des secrets GitHub pour les credentials sensibles
    \item Multi-stage builds pour optimiser les images Docker
    \item Health checks pour le monitoring des services
\end{itemize}

\section{Compétences Acquises}

\begin{table}[H]
\centering
\caption{Compétences développées}
\begin{tabular}{|l|l|}
\hline
\textbf{Domaine} & \textbf{Compétences} \\
\hline
Développement & JavaScript, React, Node.js, MongoDB \\
DevOps & Docker, GitHub Actions, CI/CD \\
Cloud & Azure App Service, ACR, Cosmos DB \\
Méthodologie & Agile, Scrum, Git \\
Architecture & REST API, Microservices, 3-tier \\
\hline
\end{tabular}
\end{table}

\section{Perspectives d'Amélioration}

\begin{itemize}
    \item \textbf{Tests} : Ajout de tests E2E avec Cypress
    \item \textbf{Monitoring} : Intégration Azure Application Insights
    \item \textbf{Temps réel} : Notifications WebSocket
    \item \textbf{Reporting} : Tableaux de bord et graphiques avancés
    \item \textbf{Kubernetes} : Migration vers Azure Kubernetes Service
    \item \textbf{Multi-environnements} : Staging et production séparés
\end{itemize}

\section{Bilan Personnel}

Ce projet m'a permis de mettre en pratique l'ensemble des connaissances acquises durant ma formation et d'acquérir une expérience concrète dans le développement d'applications modernes avec une approche DevOps.

L'intégration des pratiques CI/CD et le déploiement cloud constituent des compétences essentielles dans le monde professionnel actuel, et ce projet m'a permis de les maîtriser de manière approfondie.

% ======================================================================
%                    BIBLIOGRAPHIE
% ======================================================================

\chapter*{Bibliographie}
\addcontentsline{toc}{chapter}{Bibliographie}

\begin{enumerate}
    \item Documentation officielle MongoDB : \url{https://docs.mongodb.com/}
    \item Documentation Express.js : \url{https://expressjs.com/}
    \item Documentation React : \url{https://react.dev/}
    \item Documentation Node.js : \url{https://nodejs.org/docs/}
    \item Documentation Docker : \url{https://docs.docker.com/}
    \item Documentation GitHub Actions : \url{https://docs.github.com/en/actions}
    \item Documentation Microsoft Azure : \url{https://docs.microsoft.com/azure/}
    \item "The DevOps Handbook" - Gene Kim, Jez Humble, Patrick Debois, John Willis
    \item "Clean Code" - Robert C. Martin
    \item "Designing Data-Intensive Applications" - Martin Kleppmann
\end{enumerate}

% ======================================================================
%                    ANNEXES
% ======================================================================

\appendix

\chapter{Annexes}

\section{Commandes Azure CLI}

\begin{lstlisting}[language=bash, caption=Commandes Azure utilisées]
# Creation Cosmos DB avec Free Tier
az cosmosdb create --name flowops-mongodb \
    --resource-group flowops-rg \
    --kind MongoDB \
    --enable-free-tier true

# Configuration du conteneur
az webapp config container set \
    --name flowops-backend \
    --resource-group flowops-rg \
    --docker-custom-image-name flowopsacr.azurecr.io/flowops-backend:latest \
    --docker-registry-server-url https://flowopsacr.azurecr.io

# Variables d'environnement
az webapp config appsettings set \
    --name flowops-backend \
    --resource-group flowops-rg \
    --settings MONGO_URI="..." JWT_SECRET="..." NODE_ENV="production"
\end{lstlisting}

\section{Structure du Code Source}

\begin{lstlisting}[language=bash, caption=Arborescence du projet]
FlowOps/
|-- .github/workflows/
|   |-- ci.yml
|   |-- cd.yml
|-- backend/
|   |-- config/db.js
|   |-- controllers/
|   |-- middleware/auth.js
|   |-- models/
|   |-- routes/
|   |-- Dockerfile
|   |-- server.js
|-- frontend/
|   |-- src/
|   |   |-- components/
|   |   |-- context/
|   |   |-- pages/
|   |   |-- services/api.js
|   |-- Dockerfile
|   |-- nginx.conf
|-- docker-compose.yml
|-- README.md
\end{lstlisting}

\section{Guide d'Installation Locale}

\begin{lstlisting}[language=bash, caption=Installation locale]
# Cloner le projet
git clone https://github.com/DevITJAX/FlowOps.git
cd FlowOps

# Avec Docker Compose
docker-compose up -d

# Acces
# Frontend: http://localhost:3000
# Backend:  http://localhost:3001
\end{lstlisting}

\end{document}
